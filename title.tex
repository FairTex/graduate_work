\documentclass[14pt]{extreport}
\usepackage[utf8]{inputenc}
\usepackage[english,russian]{babel}

\usepackage{amsfonts}
\usepackage{amsmath}
\usepackage{amssymb}
\usepackage{amsthm}
\usepackage{listings}
\usepackage{amssymb,amsfonts,amsmath,mathtext,cite,enumerate,float}
\usepackage{array}
\usepackage[dvips]{graphicx}

\makeatletter
\renewcommand{\@biblabel}[1]{#1.} % Заменяем библиографию с квадратных скобок на точку:
\renewcommand{\baselinestretch}{1.5}
\makeatother

\usepackage[left=3.0cm,right=1.5cm,top=2.0cm,bottom=2.0cm,bindingoffset=0cm]{geometry}

%\renewcommand{\theenumi}{\arabic{enumi}}% Меняем везде перечисления на цифра.цифра
%\renewcommand{\labelenumi}{\arabic{enumi}}% Меняем везде перечисления на цифра.цифра
%\renewcommand{\theenumii}{.\arabic{enumii}}% Меняем везде перечисления на цифра.цифра
%\renewcommand{\labelenumii}{\arabic{enumi}.\arabic{enumii}.}% Меняем везде перечисления на цифра.цифра
%\renewcommand{\theenumiii}{.\arabic{enumiii}}% Меняем везде перечисления на цифра.цифра
%\renewcommand{\labelenumiii}{\arabic{enumi}.\arabic{enumii}.\arabic{enumiii}.}% Меняем везде перечисления на цифра.цифра

\begin{document}

   \begin{titlepage}{
        \thispagestyle{empty}
        \begin{center}
            {\footnotesize
                Министерство образования и науки Российской Федерации\\
                Федеральное государственное автономное образовательное учреждение\\
                высшего профессионального образования\\
            }

            <<Уральский федеральный университет\\
             имени первого Президента России Б.Н.Ельцина>>

             \vskip+0.5cm

            Институт математики и компьютерный наук\\
            Кафедра алгебры и дискретной математики

            \vskip+25mm

            {\bf \LARGE
                Применение нейронных сетей для калибровки оборудования на примере двух задач робототехники. \\
            }

            \vskip+15mm
        \end{center}

        \noindent\begin{parbox}[t]{9cm}{\small
                \bigskip
                \bigskip
                \bigskip
                \bigskip
                \bigskip
                \bigskip
                Допустить к защите:

                \bigskip
                \bigskip
                \bigskip

                \hbox to45mm{\hrulefill}

                \bigskip

                <<\,\hbox to10mm{\hrulefill}\,>>  \hbox to25mm{\hrulefill}  2015 г.
            }
            \end{parbox}
            \begin{parbox}[t]{9cm}{\small            
                Выпускная квалификационная \\
                работа на степень бакалавра\\
                 по направлению\\
                %НЕПРАВИЛЬНАЯ ХРЕНОТА -> 01.03.01 \\
                Математика и компьютерные науки\\
                студента группы МК-410502 \\
                Штех Геннадия Петровича\\
                \medskip
                Научный руководитель:\\
                заведующий лабораторией\\
                доктор КАКИХ?! наук \\ 
                Окуловский  \\
                Юрий Сергеевич\\
            }
            \end{parbox}

        \vfill
        \centerline{Екатеринбург}
        \centerline{2015}
        }
    \end{titlepage}

\newpage
    \tableofcontents

\newpage
    \chapter{О чем диплом?}
        Не редко при конструировнии подвижных платформ применяются двигатели, управляемые постоянным током. Характерной особенностью этих двигателей является их дешевизна и долговечность. Главная сложность при их использовании -- это сложность управления непосредственно скоростью вращения вала. Потому что фактически управлять можно только напряжением на контактах двигателя, что хорошо связано только с моментом силы. Однако при использовании двигателей для перемещения необходимо хорошо контролировать именно скорость. Поэтому остро стоит проблема определения необходимого сигнала для достижения определённой скорости. Скорость вала обычно регистрируется с помощью дополнительных устройств. В нашем случае это были энкодеры. Не буду вдаваться в принцип их работы, скажу лишь, что их точности хватало с запасом.
        Ну мы взяли нейронную сеть и попробовали решить проблемы калибровки как умеем: запихали как-то входной вектор(желаемая скорость), взяли как-то выходной(какой сигнал к этой скорости приводит), обучили сеть и обосрались: то, как мы калибровали и то, как мы использовали -- это разные сценарии.
        Основная проблема оказалась вот в чём: хуйню засунул -- хуйня выпала, мы собирали данные по следующему сценарию: выставляли сигнал, ждали, пока скорость установится, и в экспериментальную выборку добавляли пару (текущий сигнал, текущая установившаяся скорость). Понятно, что в динамике это показывало ужасающий результат. Наиболее заметен он при торможении, кривая скорости падала недостаточно быстро и робот очень сильно переезжал.
        Понятно, кажется, для начала нужно было собрать данные, соответствующие хотя бы сценарию использования. Данные стали собирать так: выставляли сигнал и все скорости, которые регистрировали, записывали в выборку обучения. К сожалению, это ни к чему не привело. Это очевидно, если взглянуть на получившуюся картинку. Множество различных данных для сети оказалось лишь шумом и улучшения не дало. (КАРТИНКА!)
        В целом жизнь устроена так: чтобы управлять штуками, нужно постоянно контроллировать, как они отреагировали на прошлое воздействие. Иначе штуки будут делать хуету. Фактически, за штуками нужно следить постоянно, с какой-то периодичностью. Поскольку жизнь вокруг штук тоже может поменяться и прежде годное управляющее воздействие прямо сейчас мчит вас в пропасть. Весь этот процесс называется циклом регуляции. Фактически, устроен он так: мы получаем обратную связь от штуки, вспоминаем, в каком состоянии нам необходимо её поддерживать и на основе обратной связи и наших грязных желаний вырабатываем управление. Возможно, даже учитываем наши предыдущие управляющие воздействия.
        В голову пришла мысль о том, что чтобы управлять динамической системой недостаточно просто построить регрессионную модель её отзывов. Оказалось, что наш двигатель -- это классическая динамическая система, требующая построения регулятора для работы. Тут мы малясь загрустили, ибо ПИД-регулятор если брать обобщенный, то там объебаться можно с коеффициентами, а если по уму его строить, то е нас такого ума нет.
        Ее мы взяли нейронную сеть и попробовали решить проблемы как умеем: запихали входной вектор(прошлый сигнал, прошлая скорость, текущий сигнал), взяли выходной(текущая скорость), собрали данных по сценарию(сценарий сцукорегулятора). Обучили сеть, а она сука поехала и стали мы счастливые очень. Фактически нейронная сеть стала инверсным нейрорегулятором, а не просто регрессионной калибровкой.


\end{document} 